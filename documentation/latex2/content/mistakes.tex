\chapter{Hibák, észrevételek}

A prototípus fejlesztés során számos helyen vétettem kisebb-nagyobb hibákat, amiket súlyosságuktól függően vagy javítani kellett, vagy valahogy kikerülni. Az első, amit meg szeretnék említeni, az a \textbf{tár kialakítása} volt. Nem voltam elég körültekintő, és anélkül terveztem meg a tárat, hogy megnéztem volna, az eredeti fegyverekben hogyan működik. Ennek eredménye lett a nagy kapacitású tár, ami aminek elméletben folyamatosan kéne adagolni a golyókat, de a gyakorlatban a szűkítésnél minden esetben elakad. Ezzel a kialakítással egyet sem sikerült tüzelni, így újra kellett tervezni a teljes tárat. Miután jobban megvizsgáltam az eredeti fegyvert, láttam, hogy a golyók egyesével sorakoznak minden esetben, ezzel gátolva az elakadást. Miután hasonlóan megterveztem az új tárat, már megbízhatóan működött, habár kisebb kapacitással, így ezt a hibát sikerült kijavítani.\\

A következő, egy szintén mechanikai természetű hiba, a \textbf{lézer modul és a kamera rögzítése}. Egyik sem állítható, és ha eredetileg nem egy pontra néznek a fegyver csövével, akkor tulajdonképpen a hiba maradandó lesz. Ez így is lett, de a kamera eltérését még a célkereszt állításával tudtam orvosolni. A lézer modul maradandóan pár fokkal félre áll, ezt sajnos el kellett fogadnom.\\

A projekt elektronikai részénél is van, amit máshogy csinálnék, ha újrakezdeném. Az első koncepcióterveknél még nem néztem utána, mekkora áramot képes felvenni a gearbox. Úgy gondoltam, hogy egy átlagos 12V-os laptoptöltőről is tud majd működni a rendszer. Így igen meglepődtem, mikor utánanéztem, mekkora teljesítményre van szüksége. Az eredmény így az lett, hogy egy külön tápot kellet adnom csak a gearbox működtetésére, ami igen kényelmetlen. Valószínűleg az lenne helyette egy jó megoldás, ha tervezek egy külön nyomtatott áramkört, amely képes arra, hogy töltsön egy akkumulátort a fegyveren. Így a gearbox arról működne, akár csak egy módosítatlan fegyveren, működésen kívül pedig a Raspberry PI tápjáról töltődne. 