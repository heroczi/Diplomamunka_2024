\pagenumbering{roman}
\setcounter{page}{1}

\selecthungarian

%----------------------------------------------------------------------------
% Abstract in Hungarian
%----------------------------------------------------------------------------
\chapter*{Kivonat}\addcontentsline{toc}{chapter}{Kivonat}

A diplomamunkám célja egy autonóm fegyverrendszer fejlesztése és elkészítése, amely funkciójában hasonlít a valós, éles helyzetben alkalmazott rendszerekhez. Ez annyit takar, hogy a fegyvernek képesnek kell egy bizonyos méretű területet belátni, ebben felismerni és azonosítani a célpontokat, majd tüzelni, vagy engedélyt kérni tüzelésre. Ezentúl szükséges funkciója a manuális vezérlés, amely segítségével az operátor valós időben tudja távolról irányítani az eszközt, egy asztali számítógép segítségével.\\

Az eszköz alkatrészei 3D nyomtatási technológiával készültek, a kötőelemeket, csapágyakat és elektronikai elemeket kivéve, első feladatom ezek nyomtatáshelyes megtervezése volt. A következő lépés a hardver és elektronikai rendszer tervezése volt, majd a szükséges modulok, szenzorok, mikrovezérlő kiválasztása és megrendelése. Az utolsó alfeladat pedig a szoftver fejlesztése, amely jelentette mind a beágyazott vezérlő szoftvert, és a számítógépen futó felhasználói felületet is.\\

A projekt megvalósítása során végig sok időt emésztett fel a rendszer tesztelése, amelyet párhuzamosan kellett végezni a későbbi alfeladatok tervezésével. Kihívást jelentett a megfelelően részletes szakirodalom felkutatása is, ugyanis a valós megoldások paramétereit gyakran nem teszik elérhetővé civilek számára, illetve a jelenleg folyó fejlesztések is titkosítottak.\\

Úgy gondolom, a téma interdiszciplináris jellegét tekintve jól illik a mechatronikai tanulmányaiba, ugyanis egyaránt érinti a műszaki mechanika,  a gyártástechnológia, a szoftverkészítés és a jelfeldolgozás területeit.\\

Diplomamunkám eredményeként megvalósítok egy működő prototípust, amit előadás keretében fogok bemutatni. Végül pedig kitérek az esetleges hibákra amiket elkövettem, a tanulságokra és a továbbfejlesztési lehetőségekre.


\vfill
\selectenglish


%----------------------------------------------------------------------------
% Abstract in English
%----------------------------------------------------------------------------
\chapter*{Abstract}\addcontentsline{toc}{chapter}{Abstract}

The aim of my thesis is to develop and build an autonomous weapon system that functions similarly to real-world systems used in live scenarios. This means that the weapon must be able to monitor a designated area, recognize and identify targets, then fire or request permission to fire. Additionally, a necessary feature is manual control, allowing the operator to remotely control the device in real-time using a desktop computer.\\

The components of the device were created using 3D printing technology, except for the fasteners, bearings, and electronic elements. My first task was to design these components to be suitable for printing. The next step was to design the hardware and electronic system, followed by selecting and ordering the necessary modules, sensors, and microcontroller. The final subtask was the development of the software, which included both the embedded controller software and the user interface running on a computer.\\

Throughout the project, significant time was consumed by system testing, which had to be carried out simultaneously with the planning of subsequent subtasks. A challenge was also posed by finding sufficiently detailed technical literature, as the parameters of real-world solutions are often not made available to civilians, and current developments are often classified.\\

I believe that the interdisciplinary nature of this topic fits well with my mechatronics studies, as it encompasses areas such as mechanical engineering, manufacturing technology, software development, and signal processing.\\

As a result of my thesis, I will produce a working prototype, which I will present during my final defense. Lastly, I will address any mistakes I made, lessons learned, and possibilities for future development.

\vfill
\selectthesislanguage

\newcounter{romanPage}
\setcounter{romanPage}{\value{page}}
\stepcounter{romanPage}