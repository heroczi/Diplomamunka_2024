\chapter{Továbbfejlesztési lehetőségek}
\textbf{Erősebb szerkezet:} Az egyik egyértelmű fejlesztési lehetőség a jobb minőségű alkatrészek használata. Elsősorban a fogaskerekekre gondolok, amelyeket ha nem 3D nyomtatással gyártottam volna, hanem valamilyen nagy pontosságú folyamattal, akkor a szerkezet nem csak csendesebb, de precízebb is lehetne.\\

\textbf{Jobb kamera:} A Raspberry PI Camera 2 megfelelő minőséggel rendelkezik arra, hogy ezt a prototípust használni tudjam vele. Azonban ha távolabbi célpontokat kellene keresni, akkor egy optikai zoom mindenképpen szükséges lenne.\\

\textbf{Vezeték nélküli működés:} A tápellátás és a kommunikáció vezeték nékülivé tétele magában hordozna új kihívásokat, például be kellene építeni egy nagyobb teljesítmény akkumulátort, illetve a vezeték nélküli kapcsolatok rendszerint lassabbak, mint a vezetékesek. Ellenben ha az eszköz célját tekintem, a távolsági operálás lenne a természetes fejlesztési irány, ahol a felhasználó akár egy másik épületből is képes lenne irányítani a fegyvert.\\

\textbf{Mozgás:} És ha már vezetékek nélkül képes működni az eszköz, akkor a következő lépés az lenne, hogy mozogni tudjon, például lánctalpak segítségével. Ez is szintén új kihívásokat jelent, de az eszköz többi részéhez képest már nem lenne nagyon bonyolult megvalósítani.\\

\textbf{Radar, távolságérzékelő:} Az eszköz következő verziójába kerülhetne akár radar is. A hasonló eszközök modern felhasználása leginkább légvédelmet jelent, azon belül drónok lelövését. Ezt vizuálisan nehézkes lehet érzékelni, hiszen igen távol is lehetnek, és kicsi a keresztmetszetük. Egy radar beépítése azonban megkönnyítené az észlelést. A távolságérzékelő pedig egyrészt több információt szolgáltat a célpontról, másrészt pedig nagy távolságoknál már számolni kell a lövedék ballisztikájával is, tehát hogy esik.
